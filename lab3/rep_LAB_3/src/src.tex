\section{Характеристики ПК студента}

Процессор Intel Core i5-10600K @ 4.10GHz, память: 16 Gb, разрядность системы: 64.

\section{Описание}

Подобрать такую эллиптическую кривую над конечным простым полем порядка p, порядок точки которой полным перебором находится за 10 минут на ПК.

\cite{elipt_cri}: \enquote{Эллиптическая кривая — это просто множество точек, описываемое уравнением:\\
$$y2 = x3 + ax + b$$}

Общий метод и алгоритм решения
Можно использовать каноническую форму эллиптической кривой:
$y2 = x3 + ax + b$

a и b пришлось подбирать случайно, а вот модуль кривой p – уже вручную, пока 
подсчёт порядка точки не стал удовлетворять условию.  


Спустя несколько итераций, было найдено значение р, удовлетворяющее условию – 29959.

Алгоритм довольно простой: сначала полным перебором находятся  находятся все точки, принадлежащие кривой, затем выбирается случайная точка и находится её порядок через сложение её самой с собой, пока сумма не станет точкой (0,0). И, разумеется, это работает за О(p^2)





Можно использовать алгоритм Шуфа (с теоремой Хассе) для ускорения решения задачи полного перебора. Сложность будет равняться О(log q), где q – число элементов поля. \\

Также есть метод комплексного умножения, он  позволяет эффективнее находить кривые с заданным количеством точек. Но плох тем, что работает лишь при определённых условиях.




\pagebreak

\section{Исходный код}


\begin{lstlisting}[language=C]
import time
import random


A = 46566389548546536960066742301497483930834559
B = 50102988429491433759123793207594191096335


def elliptic_curve(x, y, p):
    return (y ** 2) \% p == (x ** 3 + (A \% p) * x + (B \% p)) \% p


def print_curve(p):
    print("y^2 = x^3 + {0} * x + {1} (mod {2})".format(A \% p, B \% p, p))


def extended_euclidean_algorithm(a, b):
    s, old_s = 0, 1
    t, old_t = 1, 0
    r, old_r = b, a

    while r != 0:
        quotient = old_r // r
        old_r, r = r, old_r - quotient * r
        old_s, s = s, old_s - quotient * s
        old_t, t = t, old_t - quotient * t

    return old_r, old_s, old_t


def inverse_of(n, p):
    gcd, x, y = extended_euclidean_algorithm(n, p)
    assert (n * x + p * y) \% p == gcd

    if gcd != 1:
        raise ValueError(
            '{} has no multiplicative inverse '
            'modulo {}'.format(n, p))
    else:
        return x \% p


def add_points(p1, p2, p):
    if p1 == (0, 0):
        return p2
    elif p2 == (0, 0):
        return p1
    elif p1[0] == p2[0] and p1[1] != p2[1]:
        return (0, 0)

    if p1 == p2:
        s = ((3 * p1[0] ** 2 + (A \% p)) * inverse_of(2 * p1[1], p)) \% p
    else:
        s = ((p1[1] - p2[1]) * inverse_of(p1[0] - p2[0], p)) \% p
    x = (s ** 2 - 2 * p1[0]) \% p
    y = (p1[1] + s * (x - p1[0])) \% p
    return (x, -y \% p)


def order_point(point, p):
    i = 1
    check = add_points(point, point, p)
    while check != (0, 0):
        check = add_points(check, point, p)
        i += 1
    return i


if __name__ == '__main__':
    p = 29959 

    print_curve(p)
    points = []
    start = time.time()
    for x in range(0, p):
        for y in range(0, p):
            if elliptic_curve(x, y, p):
                points.append((x, y))

    cnt_points = len(points)
    print("\n")
    print("Order curve = {0}".format(cnt_points))
    point = random.choice(points)

    print("Order point {0}: {1}".format(point, order_point(point, p)))
    print("Time: {} min.".format((time.time() - start)/60))

\end{lstlisting}

\pagebreak


\section{Консоль}


\begin{alltt}
y^2 = x^3 + 18375 * x + 2871 (mod 29959)

Order curve = 30119
Order point (17186, 2225): 51476
Time: 10.278407212098439 min.

\end{alltt}



\pagebreak
\section{Выводы}
Выполнив вторую по счету (первую по порядку) лабораторную работу по курсу \enquote{Криптография}, я познакомился с новой для себя темой - факторизацией больших чисел.\\

Эта работа сама по себе не является очень трудной для выполнения, потому что для ее выполнения можно было использовать любые инструменты, но время обработки чисел, особенно n2, довольно большое, поэтому пришлось искать оптимальные инструменты для подсчета сомножителей. Так как нетривиальные сомножители n2 в любом случае будут искаться долго (время может измеряться не в часах, а в днях или неделях), было решено воспользоваться подсказкой, что в одном из вариантов какое-то число является наибольшим общим делителем моего числа n2. Я спарсил все варианты, далее каждое число и мое n2 прогнал через gcd, нашел один нетривиальный сомножитель и поделил на него n2. Это произошло очень быстро, несравнимо быстро с "честным" поиском нетривиальных сомножителей.\\

Данная лабораторная работа показывает надежность алгоритмов шифрования, таких как RSA. Потому что "честный" поиск нетривиальных сомножителей занимает очень много времени и ресурсов, но когда у нас появляется хотя бы небольшая подсказка мы можем сильно сократить время поиска.



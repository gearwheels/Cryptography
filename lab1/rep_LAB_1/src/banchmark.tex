\section{Тест производительности}
Сравним время работы алгоритма с "наивным" алгоритмом, который для каждой ячейки, находящейся в первый строчке рекурсивно высчитываем минимальный путь по каждой из трех соседних ячеек.\\ 
Время работы для матрицы. 10x10:
\begin{alltt}
Lera:l7 valeriabudnikova$ make run
./lab7 < test.txt
time_my_lab: 1e-06 s 
./time < test.txt
time_naive: 0.001899 s 
\end{alltt}
Время работы для матрицы. 15x15:
\begin{alltt}
Lera:l7 valeriabudnikova$ make run
./lab7 < test.txt
time_my_lab: 2e-06 s 
./time < test.txt
time_naive: 0.284973 s 
\end{alltt}
Как можно увидеть, мой алгоритм выиграл у наивного алгоритма, так как сложность у наивного алгоритма намного больше, из-за рекурсивного подсчета минимального пути для каждого элемента матрицы.  
\pagebreak
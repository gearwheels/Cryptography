\section{Описание}

Процесс разложения числа на его простые множетели называется факторизацией.
Для решения этой задачи существует множество алгоритмов, позволяющих находить множетели, используя свойства простых чисел.\\

Так как преподаватель не ограничил нас в выборе средств для решения задачи, я решил прибегнуть к готовому решению, онлайн калькулятору \cite{calc_page}: \enquote{Integer factorization calculator}. В этом калькуляторе реализован общий метод решета числового поля. Этот метод считается одним из самых эффективных современных алгоритмов факторизации. Он справился с поставленной
задачей для 1-го числа примерно за 2 минуту, что является впечатляющим результатом.\\
Однако 2 число имеет более 400 квадратичных знаков, факторизация которого на
обычном компьютере за разумное время практически невозможна ни одним из ныне
существующих алгоритмов. Однако я узнал что один из множетелей этого числа
определяется к наибольший общий делитель с одним из чисел другого варанта. Поэтому я быстро написал программу, пребирающую все числа других варантов, определяющую их НОД с числом моего варанта и выводящий его, если он > 1. Второе же число определяется как результат деления числа моего варанта на НОД (по свойству делителя). Для работы с большими числами и поиска делителя в этой программе я использовал библиотеку gmp.

\pagebreak

\section{Исходный код}

Программа для подбора нетривиальных сомножителей для n2.

\begin{lstlisting}[language=C]
#include <iostream>
#include <string>
#include <vector>
#include <csignal>
#include <ctime>
#include <cmath>
#include <gmpxx.h>

std::vector<mpz_class> pars (std::string inputStr){
    std::vector<mpz_class> n;
    bool strN2 = false;
    mpz_class tmp;
    tmp = 0;
    
    for(std::string::iterator i = inputStr.begin(); i != inputStr.end(); ++i){
        if (*i == '='){
            tmp = 0;
            ++i;
            while (*i != ','){ 
                tmp = tmp * 10 + (*i) - '0';
                ++i;
            }
            n.push_back(tmp);
        }
    }
    return n;
}

using namespace std;

int main(){
    mpz_class number, result;
    number = "33020229590003060461287838705174268615361274168851267464051633959211178618
2112102952744205334221107472202527866641026738964833952954682296035526349373351638998
8478563166173570125125320373840042467197427715570566783354934876492962376348888456514
9552453075463513219900132801375737362944845497774380469071477460824975475747214155932
8017841759611836896600544007093551688480761463093260025078609841325445558002874051585
8031572232760606988105994915916825321411634327591";
    vector<mpz_class> n;
    string tmpStr;
    string numStr("\0");

    while (cin >> tmpStr){
        numStr = numStr + tmpStr;
    }
    n = pars(numStr);
    for(unsigned i = 0; i < n.size(); ++i){
        mpz_gcd(result.get_mpz_t(), n[i].get_mpz_t(), number.get_mpz_t());
        if(result != 1 ){ 
            cout << "number #1: " << result << endl;
            cout << "number #2: " << number / result << endl;
            break;
        }
    }
    return 0;
}
\end{lstlisting}

\pagebreak

\section{Integer factorization calculator}

Число n1 = 5 684417 577210 707125 270927 756395 826164 432807 313246 637831 293635 062503 262393 683373\\

Нетривиальные сомножители :\\
2057 135607 937604 146710 725543 839188 722261,\\

2763 268282 011636 642475 653121 908570 867193,\\
\begin{alltt}
5 684417 577210 707125 270927 756395 826164 432807 313246 637831 293635 062503 262393 683373 (79 digits) = 2057 135607 937604 146710 725543 839188 722261 (40 digits) × 2763 268282 011636 642475 653121 908570 867193 (40 digits)

\end{alltt}



\section{Консоль}

Число n2 = 33020229590003060461287838705174268615361274168851267464051633959211178618
2112102952744205334221107472202527866641026738964833952954682296035526349373351638998
8478563166173570125125320373840042467197427715570566783354934876492962376348888456514
9552453075463513219900132801375737362944845497774380469071477460824975475747214155932
8017841759611836896600544007093551688480761463093260025078609841325445558002874051585
8031572232760606988105994915916825321411634327591\\

Нетривиальные сомножители для числа n2:\\
162257839621427704998966167419999134594347010619931431934253553\\
61581583140698584713168877872647263745674911745846902441524849327844021318\\
650107415301603729,\\

203504679139351872837215704246919986254481679672284138291298627544\\
40510486594198925640230595168104974103193039284977514197834115227998561874510\\
55561771468603595973780126474479945780171108217132313864266037113376761369990\\
84536885356359762416967611287813112363996249168423996714598121772213953355806\\
217264577079,\\

\begin{alltt}
dude@DESKTOP-545VSUH:/mnt/d/education/education/Cripta/lab1\$ g++ main.cpp -lgmpxx -lgmp
dude@DESKTOP-545VSUH:/mnt/d/education/education/Cripta/lab1\$ ./a.out <test1.txt
number \#1: 162257839621427704998966167419999134594347010619931431934253553
61581583140698584713168877872647263745674911745846902441524849327844021318
650107415301603729
number \#2: 203504679139351872837215704246919986254481679672284138291298627544
40510486594198925640230595168104974103193039284977514197834115227998561874510
55561771468603595973780126474479945780171108217132313864266037113376761369990
84536885356359762416967611287813112363996249168423996714598121772213953355806
217264577079

\end{alltt}

\pagebreak
\section{Выводы}
Выполнив четвертую лабораторную работу по курсу \enquote{Криптография}, я провел интересную исследовательскую работу.\\

Результаты данной работы были ожидаемы, наилучшие совпадения получаются, если сравнить два осмысленных текста и два текста, созданных из случайных слов, на третьем месте результат сравнения осмысленного текста и сгенерированного из случайных слов текста, на четвертом результат сравнения двух текстов, одного из случайных слов, а другого из случайных букв, и на пятом месте результат сравнения осмысленного текста и текста из случайных букв. \\

Полученные данные можно объяснить тем, что в осмысленных текстах есть правила построения предложений, какое слово за каким следуют, так как в русском языке эти правила дают некую свободу в расположении слов, результат сравнения осмысленных текстов наверно меньше, чем если бы мы использовали английскую грамматику и алфавит. Так же можно заметить, что количество знаков в алфавите, сгенерированном с помощью выбранного мною текста равно 33 знаков, что больше, чем в латинском алфавите. \\

Меня удивило, что разница в процентах между совпадениями двух осмысленных текстов и двух текстов, созданных из случайных слов, получилась небольшая, наверно это по упомянутой выше причине (о большей свободе в грамматике чем в английском).\\

В заключении могу сказать, что сложно установить длину текста, для которой сравнения будут считаться корректными, но мне кажется текст должен быть длиннее 1000 символов. 

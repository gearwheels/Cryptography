\section{Описание}

В качестве текста на осмысленном языке я выбрал поэму В.В. Маяковского "Облако в штанах".
Ее я разбил на два осмысленных текста по 7040 знаков для сравнения в программе. \\
Далее я составил словарь из слов поэмы и алфавит из букв поэмы.
В алфавит я включил все русские буквы кроме "ё", а также включил пробел, получилось 33 символа.
Из словаря я составил тексты длиной 7040 знаков из разных слов, а из алфавита я составил тексты, длина слова в которых была случайной, от 1 до длины алфавита, длина этих текстов тоже составила 7040 знаков. Все тексты, используемые в данной лабораторной работе написаны на русском языке.\\

Алгоритм сравнение достаточно прост. Подаются два текста для сравнения, далее из каждого текста берется $i$-ый элемент и сравниваются друг с другом. Если элемент совпадают, то увеличиваем счётчик совпавших символов на 1. Сравнение регистрозависимое.




\pagebreak

\section{Исходный код}
\begin{lstlisting}[language=C]
import random

def split_words(a_text):
    cur_word = ''
    prev_is_alpha = False

    for letter in a_text:
        if letter.isdigit():
            continue
        if  (letter.isalpha() and prev_is_alpha):
            cur_word += letter
        elif (letter.isalpha() and not prev_is_alpha):
            if cur_word: yield cur_word
            cur_word = letter
            prev_is_alpha = not prev_is_alpha
        else:
            if cur_word: yield cur_word
            cur_word = ''
            prev_is_alpha = False
    if cur_word: yield cur_word

def generate_random_chars(alphabet, lenght):
    ans = ""
    max_idx = len(alphabet) - 1
    for _ in range(lenght):
        ans += alphabet[random.randint(0, max_idx)]
    return ans

def generate_random_words(base, lenght):
    gen_len = 0
    ans = ""
    while gen_len < lenght:
        possible_words = list(filter(lambda x: len(x) <= lenght - gen_len, base))
        idx = random.randint(0, len(possible_words)-1)
        ans += possible_words[idx]
        gen_len += len(possible_words[idx])
        if gen_len < lenght:
            ans += " "
            gen_len += 1
    return ans

def compare_texts(text1, text2):
    if len(text1) != len(text2):
        raise ValueError
    lenght = len(text1)
    equals = 0
    for i in range(lenght):
        if text1[i] == text2[i]:
            equals += 1
    return equals / lenght

def clean_dict(old_dict):
    new_dict = set()
    for symbol in old_dict:
        if((symbol >= 'а' and symbol <= 'я') or symbol == ' '):
            new_dict.add(symbol)
    return new_dict

if __name__ == '__main__':
    filepath = "D:\\education\\education\\Cripta\\lab4\\resource\\oblako.txt"
    random.seed(42)
    oblako = ""
    with open(filepath, 'r', encoding='utf-8') as txt:
        oblako = txt.read()
    oblako_words = list(map(lambda x: x.lower(), split_words(oblako)))
    print(oblako_words[:15])
    oblako_text = " ".join(oblako_words)
    print(oblako_text[:100])
    print("\n 	Осмысленные	 тексты: \n")
    text_len = len(oblako_text) // 2
    human_text1 = oblako_text[:text_len]
    print("Длина	 полученного	 текста	 №1:", len(human_text1))
    with open("D:\\education\\education\\Cripta\\lab4\\resource\\human1.txt", 'w') as f:
        f.write(human_text1)
    human_text2 = oblako_text[text_len:2 * text_len]
    print("Длина	 полученного	 текста	 №2:", len(human_text2))
    with open("D:\\education\\education\\Cripta\\lab4\\resource\\human2.txt", 'w') as f:
        f.write(human_text2)
    print("\n Тексты	 сгенерированные	 по	 словам: \n")
    old_alphabet = list(set(oblako_text))
    alphabet = list(clean_dict(old_alphabet))
    print("Длина	 алфавита:", len(alphabet))
    chars_text1 = generate_random_chars(alphabet, text_len)
    print("Длина	 сгенерированного	 текста	 №1:", len(chars_text1))
    with open("D:\\education\\education\\Cripta\\lab4\\resource\\chars1.txt", 'w') as f:
        f.write(chars_text1)
    chars_text2 = generate_random_chars(alphabet, text_len)
    print("Длина	 сгенерированного 	текста	 №2:", len(chars_text2))
    with open("D:\\education\\education\\Cripta\\lab4\\resource\\chars2.txt", 'w') as f:
        f.write(chars_text2)
    word_base = list(set(oblako_words))
    print("\n	Тексты	 сгенерированные	 по	 буквам: \n")
    words_text1 = generate_random_words(word_base, text_len)
    print("Длина 	сгенерированного 	текста 	№1:", len(words_text1))
    with open("D:\\education\\education\\Cripta\\lab4\\resource\\words1.txt", 'w') as f:
        f.write(words_text1)
    words_text2 = generate_random_words(word_base, text_len)
    print("Длина	 сгенерированного	 текста	 №2:", len(words_text2))
    with open("D:\\education\\education\\Cripta\\lab4\\resource\\words2.txt", 'w') as f:
        f.write(words_text2)
    ans = compare_texts(human_text1, human_text2)
    print(f"\n	Доля	 совпадений	 в	 словах	 осмысленных 	текстов: {ans * 100:.2f}\% \n")
    mean = 0
    ans = compare_texts(human_text1, chars_text1)
    mean += ans
    print(f"Доля 	совпадений  	в 	словах	 осмысленного	 текста	 №1	 и	 сгенерированного	 из	 букв 	текста №1: {ans * 100:.2f}\%")
    ans = compare_texts(human_text1, chars_text2)
    mean += ans
    print(f"Доля 	совпадений 	 в	 словах	 осмысленного 	текста №1	 и 	сгенерированного 	из	 букв	 текста	 №2: {ans * 100:.2f}\%")
    ans = compare_texts(human_text2, chars_text1)
    mean += ans
    print(f"Доля	 совпадений 	 в 	словах	 осмысленного	 текста	 №2	 и 	сгенерированного	 из	 букв	 текста	 №1: {ans * 100:.2f}\%")
    ans = compare_texts(human_text2, chars_text2)
    mean += ans
    print(f"Доля 	совпадений	  в 	словах	 осмысленного 	текста 	№2	 и	 сгенерированного	 из	 букв 	текста 	№2: {ans * 100:.2f}\%")
    print(f"Средняя 	доля	 совпадений 	в	 словах: {(mean / 4) * 100:.2f}\% \n")
    mean = 0
    ans = compare_texts(human_text1, words_text1)
    mean += ans
    print(f"Доля	 совпадений 	 в	 словах 	осмысленного 	текста	 №1	  и	 сгенерированного	 из 	слов	 текста	 №1: {ans * 100:.2f}\%")
    ans = compare_texts(human_text1, words_text2)
    mean += ans
    print(f"Доля 	совпадений	 в	 словах	 осмысленного	 текста	 №1 	 и 	сгенерированного	 из	 слов	 текста	 №2: {ans * 100:.2f}\%")
    ans = compare_texts(human_text2, words_text1)
    mean += ans
    print(f"Доля 	совпадений	 в 	словах 	осмысленного 	текста	 №2 	 и	 сгенерированного 	из	 слов	 текста	 №1: {ans * 100:.2f}\%")
    ans = compare_texts(human_text2, words_text2)
    mean += ans
    print(f"Доля	 совпадений	 в	 словах 	осмысленного 	текста	 №2	  и 	сгенерированного	 из	 слов	 текста	 №2: {ans * 100:.2f}\%")
    print(f"Средняя 	доля	  совпадений 	 в 	 словах: {(mean / 4) * 100:.2f}\%\n")
    ans = compare_texts(chars_text1, chars_text2)
    print(f"Доля 	 совпадений 	 в 	 словах 	 сгенерированного	  из 	 букв 		текста 		№1	 и 	сгенерированного 	из	 букв	 текста	 №2: {ans * 100:.2f}\%")
    ans = compare_texts(words_text1, words_text2)
    print(f"Доля 	 совпадений 	 в 	 словах 	 сгенерированного 	из 	слов	 текста	 №1	 и 	сгенерированного 	из	 слов	 текста 	№2: {ans * 100:.2f}\%")

\end{lstlisting}

\pagebreak


\section{Консоль}

Судя по результатам, выведенным программой, наилучшие совпадения получаются, если сравнить два осмысленных текста и два текста, созданных из случайных слов, а на третьем месте результат сравнения осмысленного текста и сгенерированного из случайных слов текста .\\

Доля совпадений в словах осмысленных текстов: 6.34\% \\
Средняя доля совпадений в словах осмысленного текста и сгенерированного из букв текста: 2.99\%\\
Средняя доля совпадений в словах осмысленного текста и сгенерированного из слов текста: 5.68\% \\
Доля совпадений в словах текстов, сгенерированных из случайных букв: 3.29\%\\
Доля совпадений в словах текстов, сгенерированных из случайных слов: 5.78\%\\

\begin{alltt}

Осмысленные тексты: 

Длина полученного текста №1: 7030
Длина полученного текста №2: 7030

Тексты сгенерированные по словам: 

Длина алфавита: 33
Длина сгенерированного текста №1: 7030
Длина сгенерированного текста №2: 7030

Тексты сгенерированные по буквам: 

Длина сгенерированного текста №1: 7030
Длина сгенерированного текста №2: 7030

Доля совпадений в словах осмысленных текстов: 6.34\% 

Доля совпадений в словах осмысленного текста №1 и сгенерированного из букв текста №1: 2.92\%
Доля совпадений в словах осмысленного текста №1 и сгенерированного из букв текста №2: 3.09\%
Доля совпадений в словах осмысленного текста №2 и сгенерированного из букв текста №1: 3.09\%
Доля совпадений в словах осмысленного текста №2 и сгенерированного из букв текста №2: 2.89\%
Средняя доля совпадений в словах: 2.99\% 

Доля совпадений в словах осмысленного текста №1  и сгенерированного из слов текста №1: 5.73\%
Доля совпадений в словах осмысленного текста №1  и сгенерированного из слов текста №2: 5.70\%
Доля совпадений в словах осмысленного текста №2  и сгенерированного из слов текста №1: 5.46\%
Доля совпадений в словах осмысленного текста №2  и сгенерированного из слов текста №2: 5.83\%
Средняя доля совпадений в словах: 5.68\%

Доля совпадений в словах сгенерированного из букв текста №1 и сгенерированного из букв текста №2: 3.29\%
Доля совпадений в словах сгенерированного из слов текста №1 и сгенерированного из слов текста №2: 5.78\%


\end{alltt}

\pagebreak